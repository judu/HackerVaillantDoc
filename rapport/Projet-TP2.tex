\documentclass[a4paper,11pt,french]{article}
\usepackage[frenchb]{babel}
\usepackage{xunicode}
\usepackage{fontspec}
\usepackage[T1]{fontenc}




\renewcommand{\familydefault}{\sfdefault}
\title{Alma'TimeTable}
\author{Julien \textsc{Durillon} \and Alexandre \textsc{Garnier}}
\date{\today}
\begin{document}
\maketitle

\section*{Introduction}

\section{Description du service}

Le service se propose de récupérer le numéro de carte bleue d'une personne
à partir de son pseudo.

Ce service est séparé en deux sous-services~:

Le premier renvoie les informations sur une personne à partir de son
pseudo, et le deuxième renvoie le numéro de carte bleue à partir de la
personne (nom, prénom, date de naissance).


\subsection{Actions possibles}

Deux verbes sont proposés : GET et POST.

Le premier permet de récupérer une information, et le deuxième d'en
enregistrer sur le serveur.

Chacun de ces verbes a plusieurs arguments.

\subsubsection{GET}

            Ce verbe permet de récupérer deux choses différentes : une personne
            (P), et un numéro de carte bleue (CBN). Cet argument est la valeur
            du champ \og~what~\fg de la requête.

            On peut récupérer une personne à partir de son pseudo, et un numéro
            de carte bancaire, soit à partir d'une personne (P), soit à partir de
            son pseudo (N).
            
            \paragraph{Requête}
            
            Une requête GET ressemble à ceci~:
            
            \begin{verbatim}
GET
what:P|CBN
from:P|N
value:<donnée>
            \end{verbatim}
            
            
            
      \subsubsection{POST}
      
         Ce verbe permet d'enregistrer une personne, son pseudo, et son numéro
         de carte bleue.
         
         \paragraph{Requête}
         
         Une requête POST ressemble à ceci~:

         \begin{verbatim}
POST
CBN:<numéro de CB>
N:<pseudo>
P:<personne>
         \end{verbatim}

\subsection{Format des personnes}
Lors d'une requête POST, ou GET sur une personne, la personne est
encodée au format Json, entre le client et le serveur Main.

\section{Traitement des requêtes}
Selon qu'il s'agisse d'une requête GET ou d'une requête POST, le serveur
utilisera des protocoles différents pour communiquer avec les deux sous-serveurs
ConnectingPeople et GetBankId.

Dans les deux cas, le traitement de la requête passe cependant par un unique
analyseur, RequestParser, qui se charge de dispatcher le traitement des requêtes
en fonction de leur verbe.

 \subsection{Requêtes GET}
 Dans le cas d'une requête GET, c'est le protocole TCP qui est utilisé, puisque
 le serveur attend dans ce cas une réponse à la requête.
 
 Avant toute communication avec au moins un des deux sous-serveurs, un
 traitement est effectué sur les champ \og~what~\fg et \og~from~\fg~, ce afin de déduire
 des informations réclamées par le client quel(s) sous-serveur(s) sont réclamés.
 
  \subsection{Requêtes POST}
  En ce qui concerne la requête POST, puisqu'il suffit ici d'envoyer des
  informations aux deux sous-serveurs sans se soucier d'un quelconque retour, le
  protocole UDP nous est apparu tout approprié.
  
  Ici les deux sous-serveurs sont nécessairement mis à l'épreuve, puisqu'il
  s'agit d'enregistrer toutes les informations concernant une personne dans les
  bases de données de chaque serveur.

\end{document}
