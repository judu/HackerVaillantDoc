\documentclass[a4paper,11pt,french]{article}
\usepackage[frenchb]{babel}
\usepackage{xunicode}
\usepackage{fontspec}
\usepackage[T1]{fontenc}




\renewcommand{\familydefault}{\sfdefault}
\title{Alma'TimeTable}
\author{Julien \textsc{Durillon} \and Alexandre \textsc{Garnier}}
\date{\today}
\begin{document}
	\maketitle
	
	\section*{Introduction}
	
	\section{Description du service}
	
	   Le service se propose de récupérer le numéro de carte bleue d'une personne
	   à partir de son pseudo.
	   
	   Ce service est séparé en deux sous-services~:
	   
	   Le premier renvoie les informations sur une personne à partir de son
	   pseudo, et le deuxième renvoie le numéro de carte bleue à partir de la
	   personne (nom, prénom, date de naissance).
	   
	   
	   \subsection{Actions possibles}
	   
	      Deux verbes sont proposés : GET et POST.
	      
	      Le premier permet de récupérer une information, et le deuxième d'en
	      enregistrer sur le serveur.
	      
	      Chacun de ces verbes a plusieurs arguments.
	      
	      \subsubsection{GET}
	      
            Ce verbe permet de récupérer deux choses différentes : une personne
            (P), et un numéro de carte bleue (CBN). Cet argument est la valeur
            du champ «~what~» de la requête.

            On peut récupérer une personne à partir de son pseudo, et un numéro
            de carte bancaire, soit à partir d'une personne (P), soit à partir de
            son pseudo (N).
            
            \paragraph{Requête}
            
            Une requête GET ressemble à ceci~:
            
            \begin{verbatim}
GET
what:P|CBN
from:P|N
value:<donnée>
            \end{verbatim}
            
            
            
      \subsubsection{POST}
      
         Ce verbe permet d'enregistrer une personne, son pseudo, et son numéro
         de carte bleue.
         
         \paragraph{Requête}
         
         Une requête POST ressemble à ceci~:

         \begin{verbatim}
POST
CBN:<numéro de CB>
N:<pseudo>
P:<personne>
         \end{verbatim}
	
	   \subsection{Format des personnes}
	      Lors d'une requête POST, ou GET sur une personne, la personne est
	      encodée au format Json, entre le client et le serveur Main.
	
	\section{}
	
\end{document}
